\documentclass[conference]{IEEEtran}
%\IEEEoverridecommandlockouts
% The preceding line is only needed to identify funding in the first footnote. If that is unneeded, please comment it out.
\usepackage{cite}
\usepackage{amsmath,amssymb,amsfonts}
\usepackage{algorithmic}
\usepackage{graphicx}
\usepackage{textcomp}
\usepackage{xcolor}
\def\BibTeX{{\rm B\kern-.05em{\sc i\kern-.025em b}\kern-.08em
    T\kern-.1667em\lower.7ex\hbox{E}\kern-.125emX}}
\begin{document}

\title{}

\author{\IEEEauthorblockN{Tim Upthegrove}
\IEEEauthorblockA{\textit{NCT} \\
\textit{Raytheon BBN Technologies}\\
Cambridge, MA, USA \\
tim.upthegrove@raytheon.com}
}

\maketitle

\begin{abstract}
The current implementation of Internet routing is described, a critical
weakness in the design is identified, and mitigation techniques for that
weakness are introduced.
\end{abstract}

\begin{IEEEkeywords}
Internet routing, BGP, BGP hijacking, RPKI, BGPSec
\end{IEEEkeywords}

\section{Introduction}


\section{Internet Routing}
Individuals and organizations rely on Internet Service Providers (ISPs) to
direct their network traffic to the appropriate destinations.  This section
introduces how ISPs get traffic to the correct location by using the Border 
Gateway Protocol (BGP).

\subsection{Autonomous System Numbers}
Any organization that wants to own IP addresses must also be assigned an
Autonomous System Number.

\subsection{IP Prefixes}
IP addresses are typically assigned in contiguous blocks.  This is done
intentionally,

\subsection{BGP Operation}

\section{BGP Hijacking}
 * Transitive trust
 * Revisit example from introduction

\section{BGP Hijacking Mitigation Techniques}

\subsection{Filtering}

\subsection{Resource Public Key Infrastructure}

\subsection{BGPsec}

\end{document}
